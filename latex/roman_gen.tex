\documentclass[submit]{ipsj}
%\documentclass[submit,draft]{ipsj}


\usepackage[dvips]{graphicx}
\usepackage{latexsym}

\def\Underline{\setbox0\hbox\bgroup\let\\\endUnderline}
\def\endUnderline{\vphantom{y}\egroup\smash{\underline{\box0}}\\}
\def\|{\verb|}

\setcounter{巻数}{53}
\setcounter{号数}{10}
\setcounter{page}{1}

\受付{2011}{11}{4}
%\再受付{2011}{7}{16}   %省略可能
%\再再受付{2011}{7}{20} %省略可能
\採録{2011}{12}{1}

\begin{document}


\title{ローマ字変換規則DSL(仮)}

\etitle{How to Prepare Your Paper for IPSJ Journal \\
(version 2012/10/12)}

\affiliate{IPSJ}{情報処理学会\\
IPSJ, Chiyoda, Tokyo 101--0062, Japan}


\paffiliate{JU}{情報処理大学\\
Johoshori Uniersity}

\author{情報 太郎}{Joho Taro}{IPSJ}[joho.taro@ipsj.or.jp]
\author{処理 花子}{Shori Hanako}{IPSJ}
\author{学会 次郎}{Gakkai Jiro}{IPSJ,JU}[gakkai.jiro@ipsj.or.jp]

\begin{abstract}
ローマ字かな変換規則を生成するためのDSL(Domain Specific Language)を開発し
た.本論文では,変換方式の優劣を論ずるのではなく,DSLの有用性について述べる.
\end{abstract}

\begin{jkeyword}
日本語入力, ローマ字変換,
\end{jkeyword}

\begin{eabstract}
This document is a guide to prepare a draft for submitting to IPSJ
Journal, and the final camera-ready manuscript of a paper to appear in
IPSJ Journal, using {\LaTeX} and special style files.  Since this
document itself is produced with the style files, it will help you to
refer its source file which is distributed with the style files.
\end{eabstract}

\begin{ekeyword}
IPSJ Journal, \LaTeX, style files, ``Dos and Dont's'' list
\end{ekeyword}

\maketitle

%1
\section{はじめに}
キーボードからの日本語入力にはローマ字かな変換方式を利用するのが一般的である.

\subsection{漢字変換}
eeeeeeeeeeeeeeeeee

\section{おわりに}

\end{document}
